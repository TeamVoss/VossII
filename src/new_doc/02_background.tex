\section{Introduction to the Voss System}

The Voss system, hereafter referred to as Voss, started as a hardware
verification system that supported symbolic trajectory evaluation%
\index{symbolic trajectory evaluation}%
{}.
However, the main interface to the trajectory evaluator was a
general purpose functional language%
\index{functional language}%
{} with
\IX{ordered binary decision diagrams}
(\IX{OBDD}s) built in. Consequently, Boolean functions could be
represented, manipulated, and compared very efficiently. Since
these capabilities are highly desirable in formal verification
systems, it is perhaps not too surprising that Voss has become a
prototype system for various forms of verification methods. In
particular, there are now both symbolic model checkers%
\index{symbolic model checking}%
{} as well as
small theorem provers%
\index{theorem prover}%
{} written in FL---the command language of Voss.
However, since trajectory evaluation is less well known and also
the main verification methodology supported in Voss, this manual
focus mostly on this technique.

Symbolic simulation is an offspring of conventional simulation%
\index{simulation}%
{}.
Like conventional simulation%
\index{conventional simulation}%
{}, it uses a built-in model of hardware
behavior and a simulation engine%
\index{simulation engine}%
{} to compute, on demand, the behavior
of some design for some given inputs. However, it differs in that
it considers symbols rather than actual values for the design under
simulation. In this way, a symbolic simulator can simulate the
response to entire classes of values with a single simulation run.

The concept of symbolic simulation%
\index{symbolic simulation}%
{} in the context of hardware
verification was first proposed by researchers at IBM Yorktown
Heights in the late 1970's as a method for evaluating register
transfer language representations. The early programs were limited
in their analytical power since their symbolic manipulation methods
were weak. Consequently, symbolic simulation for hardware verification
did not evolve much further until more efficient methods of
manipulating symbols emerged.
The development of OBDDs for representing Boolean functions
radically transformed symbolic simulation.

Since a symbolic simulator is based on a traditional logic simulator,
it can use the same, quite accurate, electrical and timing models
to compute the circuit behavior. For example, a detailed switch-level
model, capturing charge sharing and subtle strengths phenomena,
and a timing model, capturing bounded delay assumptions, are well
within reach. Also---and of great significance---the switch-level
circuit used in the simulator can be extracted%
\index{extracted netlists}%
{} automatically from
the physical layout of the circuit. Hence, the correctness results
can link the physical layout with some higher level of specification.

The first ``post-OBDD'' symbolic simulators were simple extensions
of traditional logic simulators. In these symbolic simulators the
input values could be Boolean variables rather than only 0's, 1's
as in traditional logic simulators.  Consequently, the results of
the simulation were not single values but rather Boolean functions
describing the behavior of the circuit for the set of all possible
data represented by the Boolean variables. By representing these
Boolean functions as OBDDs the task of
comparing the results computed by the simulator and the expected
results became straightforward for many circuits. Using these
methods it has become possible to check many (combinational) circuits
exhaustively.

In~\cite{bryant-cav90,SegBry95}, Bryant and Seger began developing
a new generation of symbolic simulator based verifier.
Since the method departed quite far from traditional simulation,
they called the approach symbolic trajectory evaluation.
Here a modified version of a simulator establishes the validity
of formulas expressed in a very limited, but precisely defined, temporal logic.
This temporal logic allows the user to express properties of the circuit
over trajectories: bounded-length sequences of circuit states.
The verifier checks the validity of these formulas by a modified form
of symbolic simulation.

Although the general theory underlying symbolic trajectory evaluation,
as described in \cite{SegBry95}, is equally applicable to hardware as
software systems, we will only describe a somewhat specialized
version tailored specifically to hardware verification. For a more
comprehensive discussion of the general case, the reader is referred
to~\cite{SegBry95}.

In symbolic trajectory evaluation the circuit is modeled as operating
over logic levels%
\index{logic levels}%
{} $0$, $1$, a third level $\X$ representing an indeterminate%
\index{indeterminate value}%
{}
or unknown%
\index{unknown value}%
{} level and a fourth value $\top$ representing overconstrained%
\index{overconstrained value}%
{}
values. These values are partially ordered%
\index{partially ordered}%
{} by their ``information content%
\index{information content}%
{}'' as $\X \order 0$, $\X \order 1$, $0\order \top$, and
$1\order \top$, i.e., X conveys no information about the node value, 0 and
1 are fully defined values,
and $\top$ represent an overconstrained value or a value that is both
1 and 0 at the same time (Normally, the $\top$ value is treated as an
error condition). The only constraint placed on the circuit
model---apart from the obvious requirement that it accurately model
the physical system---is monotonicity%
\index{monotonicity}%
{} over the information ordering.
Intuitively, changing an input from $\X$ to a binary value (i.e., 0
or 1) must not cause an observed node to change from a binary value
to $\X$ or to the opposite binary value.  In extending to symbolic
simulation, the circuit nodes can take on arbitrary quaternary
(four-valued)  functions over a set of Boolean variables V. Symbolic
circuit evaluation can be thought of as computing circuit behavior
for many different operating conditions simultaneously, with each
possible assignment of 0 or 1 to the variables in V indicating a
different condition.

The biggest difference between trajectory evaluation and symbolic
simulation is the way setting nodes%
\index{setting nodes}%
{} to some value is accomplished.
In a symbolic simulator, if the user requests the system to set
the value on a node, say node A, to some value, say E, then this
node takes on this value immediately, and if the node is an input
node, keeps this value until the user requests the node to take on
another value. In trajectory evaluation, on the other hand, the
system only tries to set the node to the value E. In fact, it will
set the node to the least element in the partial order that is
consistent with both the current value on the node and the expression
E. For example, if the node currently has the value b$\X$ (i.e., if
the Boolean variable b is false, then the value on the node is 0,
otherwise it is $\X$), and we request the system to set the value on
node A to  c$\X$, then the node will in fact take on the value bc$\X$
(i.e., the node will be $\X$ unless at least one of b and c is false).
Furthermore, in trajectory evaluation, inputs do not keep their values. If the
user wants an input to a circuit to stay at 1 for 100 time units,
he or she will have to state so explicitly in the antecedent. More
about this later.

Properties of the system are expressed in a restricted form of
temporal logic%
\index{temporal logic}%
{} having just enough expressive power to describe both
circuit timing and state transition properties, but remaining simple
enough to be checked by an extension of symbolic simulation. The
basic decision algorithm checks only one basic form, the assertion%
\index{trajectory assertion}%
{},
in the form of an implication $\assert{A}{C}$; the antecedent A gives the
stimulus and current state, and the consequent C gives the desired
response and state transition. System states and stimuli are given
as trajectories%
\index{trajectory}%
{} over fixed length sequences of states.

Each of these trajectories are described with a temporal formula.
The temporal logic used here, however, is very limited. A formula
in this logic is either:
\begin{enumerate}
\item
UNC (unconstrained), 
\item
\begin{enumerate}
\item
(n is 1) (node n is equal to 1)
\item
(n is 0) (node n is equal to 0),
\end{enumerate}
\item
$F_1 \wedge F_2$ ($F_1$ and $F_2$ must both hold),
\item
$F$ when $E$ (the property represented by formula F need only hold for
those assignments satisfying the Boolean expression E),
\item
NF (F must hold in the next state).
\end{enumerate}

The temporal logic supported by the evaluator is far weaker than
that of more traditional model checkers. It lacks such basic forms
as disjunction%
\index{disjunction}%
{} and negation%
\index{negation}%
{}, along with temporal operators expressing
properties of unbounded state sequences%
\index{unbounded state sequences}%
{}. The logic was designed as
a compromise between expressive power and ease of evaluation. It
is powerful enough to express the timing and state transition
behavior of circuits, while allowing assertions to be verified by
an extended form of symbolic simulation. Note however that the
construct 4 above is very powerful.  For example, suppose one would
like to express the condition that
\[
\assert{A_1 \mbox{ "or" } A_2}{C}
\]
Clearly, this cannot be expressed directly in the logic. However,
by introducing a new Boolean variable, say a, we could rewrite the
above assertion as:
\[
\assert{(A_1 \,\mbox{when}\; a)\wedge (A_2 \,\mbox{when}\, \neg a)}{C}
\]
If we now existentially quantify%
\index{existentially quantify}%
{} away the variable $a$ from the result, 
we effectively have verified $\assert{A_1 \mbox{ "or" } A_2}{C}$.
Thus, at the cost of introducing one more Boolean variable, we can
deal with disjunction too.
However, since the number of Boolean variables used greatly affect
the efficiency of the trajectory evaluation, this aproach should
be used judiciously.

\FIG{voss}{The Voss verification system.}{voss}

The constraints placed on assertions make it possible to verify an
assertion by a single evaluation of the circuit over a number of
circuit states determined by the deepest nesting of the next-time
operators. In essence, the circuit is simulated over the unique
weakest%
\index{weakest trajectory}%
{} (in information content) trajectory allowed by the antecedent,
while checking that the resulting behavior satisfies the consequent.
In this process a Boolean function is computed expressing those
assignments for which the assertion holds.

The assertion syntax outline above is very primitive. To facilitate
generating more abstract notations, the specification language can
be embedded in a general purpose programming language. When a
program in this language is executed, it automatically can generate
the low-level temporal logic formulas and carry out the verification
process.

The Voss system%
\index{Voss system}%
{}, illustrated in \fig{voss}, consists of two different
types of programs.
First a collection of ``circuit compilers%
\index{circuit compiler}%
{}'' that translates
various hardware description languages%
\index{hardware description language}%
{} and netlists to
an internal format that is efficient to simulate.
The second part, and the main verification engine, is a program called fl.
Conceptually, fl%
\index{fl}%
{} consists of two parts as shown in \fig{fl}.
The front-end is a compiler/interpreter
for a small, fully lazy, functional language.
One of the built-in functions perform symbolic trajectory evaluation.
A specification%
\index{specification}%
{} is written as a ``program'' in this language.
When this specification program is executed, i.e., reduced to normal
form, it builds up the simulation sequence that must be run in order
to completely verify the specification.

\FIG{fl}{The FL program.}{fl}


The back-end of the Voss system is an extended symbolic simulator.
The simulator uses an externally generated finite state machine%
\index{finite state machine}%
{}
description to compute the needed trajectories. This finite-state
machine is a behavioral model of a digital circuit which can be
generated from a variety of hardware description languages.
In particular, the finite state machine can be generated from:
\begin{enumerate}
\item
a transistor netlist%
\index{transistor netlist}%
{} in .sim or .ntk format by
a suite of programs called sim2ntk%
\index{sim2ntk}%
{} and ntk2exe%
\index{ntk2exe}%
{}.
\item
a gate netlist%
\index{gate netlist}%
{} in a subset of Silos%
\index{Silos}%
{} format
by a program called silos2exe%
\index{silos2exe}%
{}.
\item
a data-flow behavioral VHDL%
\index{data-flow behavioral VHDL}%
{}%
\index{VHDL}%
{} program, a structural VHDL%
\index{structural VHDL}%
{} program, or
from an EDIF%
\index{EDIF}%
{} description, via a program called convert2fl%
\index{convert2fl}%
{}.
\end{enumerate}

Since we are using the ntk2exe tool\footnote{The ntk2exe program is
an extensive re-write of the ANAMOS%
\index{ANAMOS}%
{} tool as distributed in
the COSMOS%
\index{COSMOS}%
{} compiled switch-level simulator tool suite developed
by Randy Bryant and associates at Carnegie Mellon University.
Although virtually a complete re-write, the fundamental research ideas
embedded in ntk2exe all have their roots in the ANAMOS system.}
to pre-compile a switch-level netlist, the Voss system can carry out
switch-level verification using the full MOSSIM II%
\index{MOSSIM II}%
{} switch-level model%
\index{switch-level model}%
{}.
In addition, the finite state machine can be back-annotated with extracted
delay values%
\index{delay value}%
{} and thus fairly sophisticated delay simulation%
\index{delay simulation}%
{} can also
be carried out.

The gate level simulator is (roughly) functionally equivalent to
the SILOS II simulator%
\index{SILOS II simulator}%
{}. In addition, fairly comprehensive delay modeling
capabilities has been added for more accurate verification. In
order to achieve good performance, the symbolic simulator employs
event scheduling for both the circuit simulation as well as in
maintaining the verification conditions

Behavioral and structural VHDL is currently supported through a
translator program that is a derivative of the VHDL simulator
distributed with the Alliance 2.0%
\index{Alliance 2.0}%
{} tool suite~\cite{Alliance}.
Thus, only data-flow behavioral VHDL programs are supported.
An extensive rewrite of this part of the system is currently
underway, and a genereral release is scheduled for the June 1995.

From the Voss user's point of view the basic verification command
in the Voss system looks like:
\begin{quote}
STE options fsm weak-list ant-list cons-list trace-list;
\end{quote}
\index{STE}%
where options is a string that can give specification options to
the trajectory evaluation simulator, fsm%
\index{fsm}%
{} is a behavioral description
of a finite state machine, weak-list is a list of conditions under which
some nodes will be ``weakened%
\index{weakened}%
{}'', i.e., made more unknown, and ant-list
and cons-list denote lists of atomic constraints%
\index{constraints}%
{} used to express the
verification conditions
Each weakening element is a 4-tuple of the form (w,n,f,t), where
w denotes the Boolean condition under which this weakening of the
system model is requested, n is a node name, and f and t denote
the start and end time for the weakeneing, respectively.
Each atomic constraint is a 5-tuple of the form (b,n,v,s,f) which,
for a given trajectory, denotes the constraint that ``if the Boolean
expression b is true then the node named by n has the value v in
all states of the trajectory from the start state s up to, but not
including, the final state f''. Finally, the trace-list is a list
of triples of the form (n,f,t) requesting a trace%
\index{trace}%
{} of the node n
from time f to time t. If the verification is successful (we will
return to this in greater detail later), STE will return T (true);
otherwise it will return a boolean expression denoting the condition
under which it is valid (if it is never valid, it will simply return
F). If the verification fails for some reason, the system prints
out a counter- example for the first node for which it encounters
an incorrect value.

To give a very simple example, the command:
\begin{hol}
let ckt = load_exe "inv.exe";
STE "" ckt [] [(T, "input", F, 0, 1)] [(T, "output", T, 1, 2)] [];
\end{hol}
expresses a relationship between the input and output node of the
circuit ``inv.exe'' for one particular input value and where the
output value is delayed by one time unit.

A slightly more sophisticated approach is illustrated by the assertion:
\begin{hol}
let v = variable "v";
let ckt = load_exe "inv.exe";
STE "" ckt [] [(T, "input", v, 0, 1)] [(T, "output", (NOT v), 1, 2)] [];
\end{hol}
where the constants F and T have been replaced by the symbolic
expressions v and (NOT v).

It may appear that the temporal scope%
\index{temporal scope}%
{} of the above assertion is
limited to the first two instants of discrete time---that is, ``if
the input at time 0 is v, then the output at time 1 will be (NOT
v).'' However, the temporal scope of this assertion actually extends
infinitely along every trajectory of the finite-state machine. This
is because the automatic verification procedure considers every
state of the finite-state machine to be a possible initial state%
\index{initial state}%
{}
of the machine. At any point along any trajectory, the current
state corresponds to the initial state of some other trajectory.
Because the temporal scope of the above assertion extends infinitely
along every trajectory, the assertion can be accurately interpreted
to express the property that ``for all times t, if the value of
the input node of the inverter is v, then the value of the output
node at time t+1 will be (NOT v)''.  We will return to the pragmatics
of trajectory evaluation later in this document. For now we turn
our attention to the interface language to the Voss system.
