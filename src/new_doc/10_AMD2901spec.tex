\section{Informal specification of AMD2901}
\label{AMD2901spec}

The \IX{Am 2901} four-bit microprocessor slice
(from Advanced Micro Devices Inc.)
is a high-speed cascadable \IX{ALU} intended for use in CPUs, peripheral 
controllers and programmable microprocessors. 
The data is 4 bits wide at all points.

\subsection*{Functional Blocks}

The main functional blocks in \IX{AMD2901} are as follows:
\begin{enumerate}
\item
A 16-word by four bit two port RAM, with an up/down shifter at the input.
\begin{enumerate}
\item
Port A is an output port.
\item
Port B is a bidirectional port. 
\end{enumerate}
\item
A register (called Q) with an up/down shifter at the input
\item
An ALU source selector which select two inputs out of the following :
\begin{enumerate}
\item
Port A of RAM
\item
Port B of RAM
\item
Q register output
\item
External data input
\item
Logical 0
\end{enumerate}
\item
A 4-bit ALU, capable of performing arithmetic and logical functions on the 
selected source words.
\item
A destination selector which decides :
\begin{enumerate}
\item
whether to load the ALU output (with or without shifting) into the RAM.
\item
whether to load the ALU output (with or without shifting) or the
Q register contents (with shifting) into the Q register
\item
whether the ALU output or the Port A contents should be forwarded to
the External data output.
\end{enumerate}
\end{enumerate}

{\em NOTE:} The destination selector is not shown as an explicit block in the
figure. Its parts are included in the RAM, Q-register and output selector.

\subsection*{Inputs and Output Ports}

\begin{center}
\begin{tabular}{|r|r|c|l|} \hline
  Port & Type  &Bit  &                                                       \\
       &       &width&      Description                                       \\ \hline
     I &    in & 9   & Instruction word (discussed later)                     \\ \hline
  Aadd &    in & 4   & Address input to RAM (for READ)                       \\ \hline
  Badd &    in & 4   & Address input to RAM (for READ / WRITE )              \\ \hline
     D &    in & 4   & Data input to chip                                     \\ \hline
     Y &   out & 4   & Data output from chip                                  \\ \hline
  RAM0 & inout & 1   & Up/down shifter port connected to LSB of RAM           \\ \hline
  RAM3 & inout & 1   & Up/down shifter port connected to MSB of RAM           \\ \hline
    Q0 & inout & 1   & Up/down shifter port connected to LSB of Q-register    \\ \hline
    Q3 & inout & 1   & Up/down shifter port connected to MSB of Q-register    \\ \hline
   CLK &    in & 1   & clock                                                  \\ \hline
    C0 &    in & 1   & Carry input to ALU                                     \\ \hline
 OEbar &    in & 1   & Tri-state driver input (if this is not asserted,      \\
       &       &     &   the data output Y will be tri-stated to HIGH-Z )     \\ \hline
    C4 &   out & 1   & Carry output from ALU                                  \\ \hline
  Gbar &   out & 1   & Generate term from ALU for carry lookahead             \\ \hline
  Pbar &   out & 1   & Propagate term from ALU for carry lookahead            \\ \hline
   OVR &   out & 1   & Overflow output from ALU (this signals that an         \\
       &       &     & overflow has occurred, while performing the operation) \\ \hline
    F3 &   out & 1   & MSB of the ALU output                                  \\ \hline
   F30 &   out & 1   & Zero signal (asserted if the all 4 bits of ALU        \\
       &       &     &      output are zero).                                 \\ \hline
\end{tabular}
\end{center}

\subsection*{The Instruction Set}

The Am 2901 has a 9-bit instruction, which has three 3-bit fields whose 
functions are as follows :
\begin{enumerate}
\item
I2 downto I0: controls ALU source selector
\item
I5 downto I3: controls ALU function
\item
I8 downto I6: controls destination selector
\end{enumerate}

\subsubsection{ALU Source Operands Selected}

\begin{center}
\begin{tabular}{|c|c|c|} \hline
      Bit        &  \multicolumn{2}{c|}{ALU source} \\ 
     field       &  \multicolumn{2}{c|}{operands selected} \\ \hline
   I2 I1 I0      &       RE       &      S        \\  \hline
      000        &       A        &      Q        \\  \hline
      001        &       A        &      B        \\  \hline
      010        &       0        &      Q        \\  \hline
      011        &       0        &      B        \\  \hline
      100        &       0        &      A        \\  \hline
      101        &       D        &      A        \\  \hline
      110        &       D        &      Q        \\  \hline
      111        &       D        &      0        \\  \hline
\end{tabular}
\end{center}
{\em Note:} RE and S are the two outputs of the ALU source selector. 


\subsubsection{ALU Function}

\begin{center}
\begin{tabular}{|c|l|l|} \hline
      Bit        &  \multicolumn{2}{c|}{ALU function} \\
     field       &  \multicolumn{2}{c|}{(output \verb!-->! F)}\\ \cline{2-3}
   I5 I4 I3      &     C0 = 0      &    C0 = 1     \\ \hline
      000        &   RE + S        & RE + S + 1    \\ \hline
      001        &   S - RE - 1    & S - RE        \\ \hline
      010        &   RE - S - 1    & RE - S        \\ \hline
      011        &   RE or S       & RE or S       \\ \hline
      100        &   RE and S      & RE and S      \\ \hline
      101        &   not(RE) and S & not(RE) and S \\ \hline
      110        &   RE xor S      & RE xor S      \\ \hline
      111        &   RE xnor S     & RE xnor S     \\ \hline
\end{tabular}
\end{center}

{\em Note:} C0 is the carry-in input. F is the output of the ALU.


\subsubsection{ALU Destination}
\begin{center}
\begin{tabular}{|c|c|c|c|c|c|c|c|c|c|} \hline
    Bit   & \multicolumn{2}{c|}{RAM} & \multicolumn{2}{c|}{Q-REG.}    &        &    \multicolumn{2}{c|}{RAM}     &  \multicolumn{2}{c|}{Q-REG.} \\
   field  & \multicolumn{2}{c|}{function}  &  \multicolumn{2}{c|}{function}   &   Y    &  \multicolumn{2}{c|}{shifter}   & \multicolumn{2}{c|}{shifter} \\ \hline
 I8 I7 I6 & SHIFT & LOAD & SHIFT & LOAD &        &RAM0 & RAM3 & Q0 & Q3 \\ \hline
   000    &  ---  & ---  & ---   &   F  &    F   & --- & ---  &--- &--- \\ \hline
   001    &  ---  & ---  &  ---  & ---  &    F   & --- & ---  &--- &--- \\ \hline
   010    & ---   &  F   &  ---  & ---  &    A   & --- & ---  &--- &--- \\ \hline
   011    & ---   &  F   &  ---  & ---  &    F   & --- & ---  &--- &--- \\ \hline
   100    & down  &  F/2 & down  &  Q/2 &    F   & out &  in  &out & in \\ \hline
   101    & down  &  F/2 &  ---  & ---  &    F   & out &  in  &out &--- \\ \hline
   110    &  up   &  2F  &  up   &  2Q  &    F   &  in & out  & in &out \\ \hline
   111    &  up   &  2F  &  ---  & ---  &    F   &  in & out  &--- &out \\ \hline
\end{tabular}
\end{center}


{\em Note:}  Data that is loaded into the RAM is written at the RAM word pointed
to by the address input Badd.
Note also that the bidirectional ports are active only when some 
shifting is being done. 
Finally, whenever a bidirectional port is NOT being used as an output by
the Am2901, the Am2901 chip tristates it from its own side. Then, 
it can be used as an input from the external world or it may be 
left inactive. 
